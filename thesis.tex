\documentclass[11pt, oneside]{article}   	% use "amsart" instead of "article" for AMSLaTeX format
\usepackage{geometry}                		% See geometry.pdf to learn the layout options. There are lots.
\geometry{letterpaper}                   		% ... or a4paper or a5paper or ... 
%\geometry{landscape}                		% Activate for rotated page geometry
%\usepackage[parfill]{parskip}    		% Activate to begin paragraphs with an empty line rather than an indent
\usepackage{graphicx}				% Use pdf, png, jpg, or eps§ with pdflatex; use eps in DVI mode
								% TeX will automatically convert eps --> pdf in pdflatex		
\usepackage{amssymb}

%2SetFonts

%SetFonts


\title{Expanding the Modeling of Type Ia Supernovae}
\author{Abigail Bishop}
%\date{}							% Activate to display a given date or no date

\begin{document}
\maketitle

% Something about an Abstract here.

\section{Introduction}

  % What is a Supernovae, specifically Type Ia and define subchandra
  
  % Why is it important for astronomy and why should we simulate it
  
  % How are these things typically simulated
  
  Supernovae (SNe), the explosive deaths of stars, are some of the most spectacular events in the universe. A special kind of supernovae, the Type Ia Supernovae, is especially notable for its use as an astronomical measure of distance. Type Ia supernovae occur when two stars are in orbit around each other. At least one star has already died once and has become a white dwarf (WD) usually with a mass around that of our sun, called a solar mass ($M_{\odot}$). Material in the outer shells of the WD's partner can be stolen by the WD through its gravitational field. Once the WD has taken enough material to reach a mass of 1.4 times $M_{\odot}$, it explodes. This 1.4$M_{\odot}$ is called the Chandrasekhar mass. The explosion releases light of many colors and other wavelengths. The graph plotting how bright each wavelength of light is in the explosion is the spectrum. Since these stars usually have the same mass and elemental composition when they explode, the explosions have the same brightness and uniquely Type Ia SNe spectrum. 
  
  An observational astronomer can observe an event resembling a Type Ia SNe and use its spectrum to confirm it is a Type Ia Supernovae. The astronomer can then take the brightness of the explosion and use it to calculate how far away the supernovae is. By using Type Ia SNe in other galaxies to measure how far away they are and other measurements, observational cosmologists have been able to measure the rate at which our universe is expanding. Without accurate models of Type Ia SNe, these calculations would not be possible. 
  
  The dilemma is that not all Type Ia Supernovae wait until their mass reaches  the Chandrasekhar mass of 1.4$M_{\odot}$ before they explode. Some explode at lower masses, in the subChandra range where their mass is less than 1.4$M_{\odot}$. This means the overall brightness and the spectrum of the explosion is slightly different from the Chandrasekhar mass Type Ia Supernovae. If an observational astronomer uses a subChandra Type Ia supernovae to measure distance, they will calculate a larger distance if they assume it is a Chandrasekhar mass star. This means, as computational nuclear astrophysicists, we need to model subChandra Type Ia SNe to help the observational astronomers make accurate calculations of astronomical distance. This has been done before, however, a paper by Ken Shen and Lars Bildsten has pointed out that previous simulations have not included as many possible nuclear reactions as it should have to make the calculations accurate. In this thesis I discuss the methods by which we use a computer code to simulate Type Ia SNe with the elements suggested by Shen and Bildsten. 

\section{Codes}

  \subsection{GitHub}
  	
    GitHub is a website used to mediate code sharing and code collaboration. Most of the codes discussed in this report are available online, with all code available for viewing, editing, and using. This is the concept behind open-sourced software. 

  \subsection{Pynucastro}
    
    Pynucastro is an open-sourced coding package developed by Dr. Michael Zingale and Dr. Donald Willcox to create networks of nuclear reactions that can be used in their simulations. %Cite Pynucastro paper
    It does so by collecting information on the rate at which nuclear reactions occur at different temperatures from a database called AMREX. The formatted reaction rates are compiled into a network that can be access by simulations like Microphysics and Castro. 
  
  \subsection{Microphysics}
  
    Microphysics is an open-sourced software, available on the StarKiller GitHub repository, with routines designed to incorporate aforementioned networks of reaction rates into astrophysical simulations. % Cite Microphysics
    It is equipped with unit tests designed to run stellar simulations in a short amount of time to troubleshoot problems on a small scale before promoting the simulations to a larger scale operation. Microphysics comes with some preexisting networks but can also use new networks generated by Pynucastro. 
  
  \subsection{Castro}
  
    Castro is an open sourced software, available under the AMReX-Astro repository on GitHub, and is designed to run full scale simulations of astrophysical events in one dimension, two dimensions, and three dimensions. Simulations include Type Ia SNe, core collapse SNe, and WD mergers. 
  
\section{Motivation}
  
  As previously mentioned, Type Ia SNe are critical to measuring large astronomical distances. Observational astronomers rely heavily on accurate estimations for what they observe to make scientific claims to the data. In a 2016 paper by Lars Bildsten and Kenneth Shen, they theorized that the three elements used to simulate Type Ia supernovae was not enough to accurately simulate the spectra and light curve of a Type Ia supernovae. %Cite Shen and Bildsten
  Additionally, not all Type Ia SNe occur when the WD becomes a total of 1.4 $M_{\odot}$
  
  We have simulations of 
  
  
  
\section{Simulation Results}

  \subsection{Pynucastro}
    
  
  \subsection{Microphysics}
  
  
  \subsection{Castro}
  
  
\section{Conclusion}

  
\end{document}  